\documentclass{beamer}
\usetheme{Madrid}
\usecolortheme{default}

\usepackage{amsmath}
\usepackage{booktabs}
\usepackage{xcolor}

\definecolor{good}{RGB}{40,167,69}
\definecolor{bad}{RGB}{220,53,69}

\title{Fantasy Baseball Draft Tool}
\subtitle{How Player Values Are Calculated}
\author{Kevin Kuruc}
\date{February 2026}

\begin{document}

\begin{frame}
\titlepage
\end{frame}

\begin{frame}{Outline}
\tableofcontents
\end{frame}

%=============================================================================
\section{The Core Idea}
%=============================================================================

\begin{frame}{What Are We Trying to Do?}
\textbf{Goal:} Rank players by how much they help you win H2H category matchups.

\vspace{0.5cm}
\textbf{League format:} 14-category H2H weekly
\begin{itemize}
    \item Hitting: R, HR, RBI, SB, SO, TB, OBP
    \item Pitching: W, SV, K, HLD, ERA, WHIP, QS
\end{itemize}

\vspace{0.5cm}
\textbf{Key insight:} Not all stats are created equal. A steal is worth more than a strikeout because steals are \textit{tighter}---there's less variance week-to-week.
\end{frame}

\begin{frame}{The Win Probability Formula}
Each week, your team produces some total in each category. So does your opponent.

\vspace{0.3cm}
If we assume weekly totals are normally distributed:
\[
P(\text{win category}) = \Phi\left(\frac{\mu_{\text{you}} - \mu_{\text{opponent}}}{\sigma \cdot \sqrt{2}}\right)
\]

\vspace{0.3cm}
\textbf{Where:}
\begin{itemize}
    \item $\Phi$ = the bell curve CDF (converts z-scores to probabilities)
    \item $\sigma$ = how much that category varies week-to-week
    \item The $\sqrt{2}$ accounts for both teams having variance
\end{itemize}
\end{frame}

\begin{frame}{Why Does Standard Deviation Matter?}
\textbf{Tight categories} (low SD) are more predictable.

A small edge goes a long way.

\vspace{0.5cm}
\textbf{Wide categories} (high SD) are noisy.

Even a big edge might not matter---luck dominates.

\vspace{0.5cm}
\begin{center}
\begin{tabular}{lrrl}
\toprule
Category & SD & 1/SD & Interpretation \\
\midrule
SB & 2.57 & 0.39 & \textcolor{good}{High leverage} \\
SV & 1.54 & 0.65 & \textcolor{good}{High leverage} \\
K (pitching) & 11.79 & 0.08 & \textcolor{bad}{Low leverage} \\
TB & 15.94 & 0.06 & \textcolor{bad}{Low leverage} \\
\bottomrule
\end{tabular}
\end{center}
\end{frame}

%=============================================================================
\section{Marginal Value}
%=============================================================================

\begin{frame}{What Is Marginal Value?}
\textbf{Marginal value} = how much a player improves your expected wins \textit{compared to a replacement-level player}.

\vspace{0.5cm}
Think of it this way:
\begin{enumerate}
    \item Start with your current roster
    \item Calculate your win probability in each category
    \item Add a player to an empty slot
    \item Recalculate win probabilities
    \item Marginal value = new total $-$ old total
\end{enumerate}

\vspace{0.3cm}
If your roster is empty, you're comparing to a team of 9 replacement-level hitters.
\end{frame}

\begin{frame}{What Is Replacement Level?}
\textbf{Replacement level} = the production you could get for free (waiver wire).

\vspace{0.5cm}
We estimate this by:
\begin{enumerate}
    \item Ranking all projected hitters
    \item Taking players ranked 155--175 (just beyond draft pool)
    \item Averaging their per-PA production
\end{enumerate}

\vspace{0.5cm}
\textbf{Replacement hitter (600 PA):}
\begin{center}
\begin{tabular}{lrrrrrrr}
\toprule
R & HR & RBI & SO & TB & SB & OBP \\
\midrule
73 & 20 & 72 & 134 & 224 & 9 & .320 \\
\bottomrule
\end{tabular}
\end{center}
\end{frame}

%=============================================================================
\section{Worked Example: Jos\'{e} Ram\'{i}rez}
%=============================================================================

\begin{frame}{Setup: Empty Roster}
Imagine you have no players drafted yet.

\vspace{0.3cm}
Your 9 hitter slots are filled with replacement-level production:

\begin{center}
\begin{tabular}{lrr}
\toprule
Category & Your Team (weekly) & League Avg \\
\midrule
R & 26.3 & 29.0 \\
HR & 7.2 & 8.0 \\
RBI & 25.9 & 27.9 \\
SO & 48.2 & 50.1 \\
TB & 80.6 & 88.9 \\
SB & 3.2 & 4.7 \\
OBP & .320 & .320 \\
\bottomrule
\end{tabular}
\end{center}

You're below average in most categories, but slightly better in SO (fewer strikeouts is good).
\end{frame}

\begin{frame}{Win Probabilities: Replacement Team}
Using the formula, here's your chance of winning each category:

\begin{center}
\begin{tabular}{lrr}
\toprule
Category & Diff vs Avg & P(win) \\
\midrule
R & $-2.7$ & 37.7\% \\
HR & $-0.8$ & 42.2\% \\
RBI & $-1.9$ & 41.9\% \\
SO & $+1.9$ & \textcolor{good}{57.0\%} \\
TB & $-8.2$ & 35.8\% \\
SB & $-1.5$ & 34.0\% \\
OBP & $0$ & 50.0\% \\
\midrule
\textbf{Total} & & \textbf{2.99 / 7} \\
\bottomrule
\end{tabular}
\end{center}

A replacement-level team expects to win about 3 of 7 hitting categories.
\end{frame}

\begin{frame}{Enter Jos\'{e} Ram\'{i}rez}
\textbf{Ram\'{i}rez's projected line:}\\
679 PA, 98 R, 30 HR, 94 RBI, 78 SO, 300 TB, 34 SB, .348 OBP

\vspace{0.5cm}
\textbf{Weekly contribution vs replacement:}
\begin{center}
\begin{tabular}{lrrr}
\toprule
Stat & Ram\'{i}rez & Replacement & Diff \\
\midrule
R/wk & 3.92 & 2.92 & \textcolor{good}{+1.00} \\
HR/wk & 1.20 & 0.80 & \textcolor{good}{+0.40} \\
RBI/wk & 3.76 & 2.88 & \textcolor{good}{+0.88} \\
SO/wk & 3.12 & 5.36 & \textcolor{good}{$-2.24$} \\
TB/wk & 12.00 & 8.96 & \textcolor{good}{+3.04} \\
SB/wk & 1.36 & 0.36 & \textcolor{good}{+1.00} \\
OBP & .348 & .320 & \textcolor{good}{+.028} \\
\bottomrule
\end{tabular}
\end{center}

Ram\'{i}rez is better than replacement in \textit{every} category.
\end{frame}

\begin{frame}{New Win Probabilities}
With Ram\'{i}rez replacing one replacement hitter:

\begin{center}
\begin{tabular}{lrrr}
\toprule
Category & Before & After & Gain \\
\midrule
R & 37.7\% & 42.2\% & +4.5\% \\
HR & 42.2\% & 46.0\% & +3.8\% \\
RBI & 41.9\% & 45.6\% & +3.6\% \\
SO & 57.0\% & 65.2\% & \textcolor{good}{+8.1\%} \\
TB & 35.8\% & 40.9\% & +5.1\% \\
SB & 34.0\% & 44.5\% & \textcolor{good}{+10.5\%} \\
OBP & 50.0\% & 52.2\% & +2.2\% \\
\midrule
\textbf{Total} & 2.99 & 3.36 & \textbf{+0.378} \\
\bottomrule
\end{tabular}
\end{center}

\textbf{Ram\'{i}rez's marginal value: 0.378}

He adds 0.378 expected category wins per week.
\end{frame}

\begin{frame}{Why Do SB and SO Dominate?}
Ram\'{i}rez's biggest gains:
\begin{itemize}
    \item \textbf{SB: +10.5\%} from just 1 extra steal/week
    \item \textbf{SO: +8.1\%} from 2.24 fewer strikeouts/week
\end{itemize}

\vspace{0.3cm}
Compare to TB: +5.1\% from +3.04 TB/week.

\vspace{0.5cm}
\textbf{The math:}
\begin{itemize}
    \item SB: $1.00 \div 2.57 = 0.39$ standard deviations
    \item TB: $3.04 \div 15.94 = 0.19$ standard deviations
\end{itemize}

\vspace{0.3cm}
One steal moves the needle \textbf{twice as much} as three total bases, because steals have a tighter distribution.
\end{frame}

%=============================================================================
\section{Relief Pitcher Valuation}
%=============================================================================

\begin{frame}{The Saves vs Holds Tradeoff}
Elite closers get saves but sacrifice holds.

\vspace{0.3cm}
\textbf{Replacement RP (per week):}
\begin{center}
\begin{tabular}{lr}
\toprule
Stat & Replacement \\
\midrule
SV/wk & 0.12 \\
HLD/wk & 0.85 \\
K/wk & 2.69 \\
\bottomrule
\end{tabular}
\end{center}

\vspace{0.3cm}
Replacement RPs are \textbf{middle relievers}---they get holds, not saves.

An elite closer gives up ~0.75 holds/week to gain ~1.35 saves/week.

Is that trade worth it?
\end{frame}

\begin{frame}{Saves Have High Leverage}
\begin{center}
\begin{tabular}{lrr}
\toprule
Category & SD & 1/SD \\
\midrule
SV & 1.54 & \textbf{0.65} \\
HLD & 1.64 & 0.61 \\
K & 11.79 & 0.08 \\
\bottomrule
\end{tabular}
\end{center}

\vspace{0.3cm}
Saves and holds have similar SDs, but closers gain more saves than they lose in holds.

\vspace{0.3cm}
\textbf{Net effect:} Elite closers are valuable because the save gain ($+1.35$/wk) exceeds the hold loss ($-0.75$/wk) in absolute terms.
\end{frame}

\begin{frame}{Why ERA/WHIP Don't Matter Much for RPs}
ERA and WHIP are \textbf{innings-weighted}.

\vspace{0.3cm}
A typical team pitches ~40 IP/week:
\begin{itemize}
    \item 5 SP $\times$ 6.5 IP = 32.5 IP
    \item 3 RP $\times$ 2.5 IP = 7.5 IP
\end{itemize}

\vspace{0.3cm}
A reliever contributes \textbf{6\%} of team innings.

\vspace{0.3cm}
Even if an elite RP has much better ERA than replacement (3.06 vs 3.50), the team ERA only improves by $\approx$0.02 points.

\vspace{0.3cm}
\textbf{Conclusion:} RPs earn their value through saves, holds, and strikeouts---not ERA/WHIP.
\end{frame}

%=============================================================================
\section{Comparing Projection Systems}
%=============================================================================

\begin{frame}{Three Projection Systems}
The tool supports three projection sources:

\begin{enumerate}
    \item \textbf{Depth Charts} -- FanGraphs composite (default for playing time)
    \item \textbf{The Bat} -- Tom Tango's system
    \item \textbf{The BatX} -- Extended version of The Bat
\end{enumerate}

\vspace{0.5cm}
All three use:
\begin{itemize}
    \item Same playing time (normalized to Depth Charts)
    \item Same replacement level
    \item Same weekly SDs
\end{itemize}

\vspace{0.3cm}
\textbf{Only difference:} Skill estimates (HR rate, SB rate, K\%, OBP, etc.)
\end{frame}

\begin{frame}{PA Normalization}
Different systems project different playing time.

\vspace{0.3cm}
\textbf{Problem:} We want to compare skill, not PT estimates.

\vspace{0.3cm}
\textbf{Solution:} Scale all systems to use Depth Charts PA.

\vspace{0.5cm}
Example:
\begin{itemize}
    \item The Bat projects Player X at 500 PA, 25 HR
    \item Depth Charts projects Player X at 600 PA
    \item Normalized: 600 PA, 30 HR (scaled up by 600/500)
\end{itemize}

\vspace{0.3cm}
Rate stats (OBP, K\%) stay unchanged---they reflect skill, not volume.
\end{frame}

%=============================================================================
\section{Low-PA Players}
%=============================================================================

\begin{frame}{The Problem with Part-Time Players}
A player projected for 400 PA will have lower counting stats than a 600 PA player, even if they're equally skilled per-PA.

\vspace{0.5cm}
\textbf{Solution:} Supplement everyone to 600 PA with replacement-level production.

\vspace{0.5cm}
\textbf{Example:} Player with 400 PA, 60 runs, .350 OBP
\begin{itemize}
    \item Gap: 200 PA at replacement level
    \item Replacement runs: $200 \times 0.121 = 24$
    \item Total runs: $60 + 24 = 84$
    \item Blended OBP: $(400 \times .350 + 200 \times .320) / 600 = .340$
\end{itemize}

\vspace{0.3cm}
This answers: ``What would this player produce over a full roster slot?''
\end{frame}

%=============================================================================
\section{Key Takeaways}
%=============================================================================

\begin{frame}{Summary}
\textbf{1. Marginal value measures wins added vs replacement.}

Not raw stats---wins.

\vspace{0.3cm}
\textbf{2. Tight categories (low SD) have high leverage.}

SB, SV, HLD matter more per unit than TB, K.

\vspace{0.3cm}
\textbf{3. Replacement level is the baseline.}

Players ranked 155-175 define what's ``free.''

\vspace{0.3cm}
\textbf{4. Rate stats are diluted across rosters.}

OBP is shared by 9 hitters. ERA/WHIP are innings-weighted.

\vspace{0.3cm}
\textbf{5. Elite closers win the saves/holds tradeoff.}

They give up holds but gain more in saves.
\end{frame}

\begin{frame}{Category Leverage Reference}
\begin{center}
\begin{tabular}{lrrr}
\toprule
Category & SD & Mean & 1/SD \\
\midrule
\multicolumn{4}{l}{\textbf{Hitting}} \\
SB & 2.57 & 4.74 & \textcolor{good}{0.39} \\
HR & 2.93 & 8.02 & 0.34 \\
R & 6.03 & 28.96 & 0.17 \\
RBI & 6.72 & 27.86 & 0.15 \\
SO & 7.45 & 50.11 & 0.13 \\
TB & 15.94 & 88.87 & 0.06 \\
\midrule
\multicolumn{4}{l}{\textbf{Pitching}} \\
SV & 1.54 & 2.27 & \textcolor{good}{0.65} \\
HLD & 1.64 & 2.30 & 0.61 \\
L & 1.83 & 3.08 & 0.55 \\
K & 11.79 & 50.90 & 0.08 \\
\bottomrule
\end{tabular}
\end{center}

Higher 1/SD = more leverage per unit.
\end{frame}

\end{document}
