\documentclass{beamer}
\usetheme{Madrid}
\usecolortheme{default}

\usepackage{amsmath}
\usepackage{booktabs}
\usepackage{xcolor}

% Custom colors for uncertainty highlighting
\definecolor{uncertain}{RGB}{220,53,69}
\definecolor{confident}{RGB}{40,167,69}

\newcommand{\uncertain}[1]{\textcolor{uncertain}{\textbf{[?] #1}}}
\newcommand{\confident}[1]{\textcolor{confident}{#1}}

\title{Fantasy Baseball Draft Tool}
\subtitle{Documentation and Methodology}
\author{Claude + Kevin}
\date{February 2026}

\begin{document}

\begin{frame}
\titlepage
\end{frame}

\begin{frame}{Outline}
\tableofcontents
\end{frame}

%=============================================================================
\section{System Overview}
%=============================================================================

\begin{frame}{What Does This Tool Do?}
\textbf{Goal:} Rank players by their marginal contribution to winning H2H categories.

\vspace{0.5cm}
\textbf{Key Components:}
\begin{enumerate}
    \item \texttt{create\_league\_stats.py} -- Generates hitter CSVs with z-scores
    \item \texttt{create\_pitching\_stats.py} -- Generates pitcher CSVs
    \item \texttt{normalize\_pa.py} -- Aligns PA between projection systems
    \item \texttt{draft\_tool.html} -- Interactive draft UI with real-time valuations
\end{enumerate}

\vspace{0.5cm}
\textbf{League Format:} 14-category H2H weekly
\begin{itemize}
    \item Hitting: R, HR, RBI, SB, SO, TB, OBP
    \item Pitching: W, SV, K, HLD, ERA, WHIP, QS
\end{itemize}
\end{frame}

\begin{frame}{The Core Idea: Marginal Win Probability}
In H2H, you win a category if your weekly total beats your opponent's.

\vspace{0.3cm}
Assuming team totals are normally distributed:
\[
P(\text{win category}) = \Phi\left(\frac{\mu_{\text{my team}} - \mu_{\text{opponent}}}{\sigma \cdot \sqrt{2}}\right)
\]

where:
\begin{itemize}
    \item $\Phi$ = standard normal CDF
    \item $\sigma$ = weekly standard deviation for that category (from 2024 data)
    \item The $\sqrt{2}$ comes from $\text{Var}(X - Y) = \text{Var}(X) + \text{Var}(Y) = 2\sigma^2$
\end{itemize}

\vspace{0.3cm}
\textbf{A player's value} = how much they move $P(\text{win})$ vs. replacement.
\end{frame}

\begin{frame}{Weekly Standard Deviations (2024 League Data)}
\textbf{Hitting Categories:}
\begin{center}
\begin{tabular}{lrrrr}
\toprule
Category & SD & Mean & CV & 1/SD \\
\midrule
R & 6.03 & 28.96 & 21\% & 0.17 \\
HR & 2.93 & 8.02 & 37\% & 0.34 \\
RBI & 6.72 & 27.86 & 24\% & 0.15 \\
SB & 2.57 & 4.74 & \textbf{54\%} & \textbf{0.39} \\
SO & 7.45 & 50.11 & 15\% & 0.13 \\
TB & 15.94 & 88.87 & 18\% & 0.06 \\
OBP & 0.04 & 0.320 & 13\% & 25.0 \\
\bottomrule
\end{tabular}
\end{center}

\vspace{0.2cm}
\textbf{Two metrics:}
\begin{itemize}
    \item \textbf{1/SD}: Marginal value per unit (used for player valuation)
    \item \textbf{CV = SD/Mean}: How luck-dependent the category is
\end{itemize}
\end{frame}

\begin{frame}{SD vs Coefficient of Variation}
\textbf{Why does absolute SD matter for marginal value?}

The win probability formula is:
\[
P(\text{win}) = \Phi\left(\frac{\mu_{me} - \mu_{opp}}{\sigma \sqrt{2}}\right)
\]

Adding +1 unit shifts numerator by +1. Denominator is $\sigma\sqrt{2}$.

\textbf{The mean doesn't appear} -- only absolute SD matters for marginal impact.

\vspace{0.3cm}
\textbf{But CV tells us something different:}
\begin{itemize}
    \item SB has CV = 54\% $\rightarrow$ outcomes highly variable relative to baseline
    \item SO has CV = 15\% $\rightarrow$ outcomes more predictable
\end{itemize}

\vspace{0.3cm}
High CV means the category is ``noisy,'' but the marginal value of +1 unit is still determined by 1/SD.
\end{frame}

\begin{frame}{Pitching Standard Deviations}
\begin{center}
\begin{tabular}{lrrrr}
\toprule
Category & SD & Mean & CV & 1/SD \\
\midrule
L & 1.83 & 3.08 & 59\% & 0.55 \\
SV & 1.54 & 2.27 & \textbf{68\%} & \textbf{0.65} \\
K & 11.79 & 50.90 & 23\% & 0.08 \\
HLD & 1.64 & 2.30 & \textbf{71\%} & 0.61 \\
ERA & 1.31 & 3.79 & 35\% & 0.76 \\
WHIP & 0.21 & 1.20 & 17\% & 4.76 \\
QS & 1.40 & 3.25 & 43\% & 0.71 \\
\bottomrule
\end{tabular}
\end{center}

\vspace{0.3cm}
SV and HLD have \textbf{both} high leverage (high 1/SD) \textbf{and} high noise (high CV).

These categories are volatile but also where edges compound.
\end{frame}

%=============================================================================
\section{Z-Score Calculation (CSV Generation)}
%=============================================================================

\begin{frame}{How Z-Scores Are Calculated in CSVs}
For \textbf{counting stats} (R, HR, RBI, TB, SB):
\[
z = \frac{\text{Season Total} / 25}{\text{Weekly SD}}
\]

For \textbf{strikeouts} (lower is better):
\[
z_{SO} = -\frac{\text{Season SO} / 25}{\text{SD}_{SO}}
\]

\vspace{0.3cm}
For \textbf{OBP} (rate stat):
\[
z_{OBP} = \frac{(\text{OBP} - 0.320)}{9 \times \text{SD}_{OBP}}
\]

\vspace{0.3cm}
\textbf{Why divide by 9?} OBP is a rate stat. A roster has 9 hitters, so one player's OBP contributes $\frac{1}{9}$ of team OBP. This scales OBP's contribution appropriately.
\end{frame}

\begin{frame}{Total Z-Score}
\[
z_{\text{total}} = z_R + z_{HR} + z_{RBI} + z_{SB} + z_{TB} + z_{SO} + z_{OBP}
\]

\textbf{This is a simplified ranking metric} used to sort players in the CSV.

\vspace{0.5cm}
\textbf{Important distinction:}
\begin{itemize}
    \item \textbf{CSV z-scores}: Quick approximation for ranking
    \item \textbf{Draft tool marginal value}: Exact calculation considering current roster
\end{itemize}

\vspace{0.3cm}
Both should \textit{mostly} agree on rankings, but the draft tool is more precise because it accounts for your specific team composition.
\end{frame}

%=============================================================================
\section{Replacement Level}
%=============================================================================

\begin{frame}{What Is Replacement Level?}
\textbf{Definition:} The production available from freely available players (waiver wire).

\vspace{0.3cm}
\textbf{Methodology:}
\begin{enumerate}
    \item Rank all hitters by $z_{\text{total}}$ using Depth Charts projections
    \item Take players ranked 155--175 (just beyond 16 teams $\times$ 9 hitters = 144)
    \item Average their per-PA rates
\end{enumerate}

\vspace{0.3cm}
\textbf{The cohort (21 players):}\\
{\small Spencer Steer, Miguel Andujar, Ezequiel Tovar, Jonathan Aranda, Addison Barger, Nathan Lukes, Kyle Manzardo, Colt Keith, Josh Lowe, Romy Gonzalez, Samuel Basallo, Francisco Alvarez, Lars Nootbaar, Joey Ortiz, Tyler O'Neill, Ryan O'Hearn, Victor Robles, Jake Fraley, JJ Bleday, Munetaka Murakami, Chase Meidroth}
\end{frame}

\begin{frame}{Replacement Level Per-PA Rates}
\begin{center}
\begin{tabular}{lrrr}
\toprule
Stat & Per-PA Rate & Full Season (600 PA) & Weekly \\
\midrule
R & 0.1212 & 73 & 2.92 \\
HR & 0.0330 & 20 & 0.80 \\
RBI & 0.1208 & 72 & 2.88 \\
SO & 0.2226 & 134 & 5.36 \\
TB & 0.3725 & 224 & 8.96 \\
SB & 0.0152 & 9 & 0.36 \\
OBP & 0.320 & -- & -- \\
\bottomrule
\end{tabular}
\end{center}

\vspace{0.3cm}
\textbf{OBP is hardcoded to 0.320} (league average) because the cohort's actual OBP (0.324) exceeded league average. This would make replacement players OBP-positive, which is conceptually wrong---replacement should be neutral, not additive.
\end{frame}

\begin{frame}{PA Supplementation}
\textbf{Problem:} Low-PA players look bad in counting stats.

\vspace{0.3cm}
\textbf{Solution:} Supplement everyone to 600 PA with replacement-level production.

\vspace{0.3cm}
For a player with 450 PA:
\begin{align*}
\text{Gap PA} &= 600 - 450 = 150 \\
\text{Runs} &= \text{Projected Runs} + 150 \times 0.1212 \\
\text{OBP} &= \frac{450 \times \text{proj OBP} + 150 \times 0.320}{600}
\end{align*}

\vspace{0.3cm}
\textbf{Interpretation:} ``What would this player produce if they played full-time, with a replacement-level player filling in the rest?''
\end{frame}

%=============================================================================
\section{Detailed Example: Jos\'{e} Ram\'{i}rez}
%=============================================================================

\begin{frame}{Setup: A Team of Replacement Players}
Consider a team with 9 replacement-level hitters.

\vspace{0.3cm}
\textbf{Each replacement hitter's weekly production:}
\begin{center}
\begin{tabular}{lr}
\toprule
Stat & Per Hitter/Week \\
\midrule
R & 2.92 \\
HR & 0.80 \\
RBI & 2.88 \\
SO & 5.36 \\
TB & 8.96 \\
SB & 0.36 \\
OBP & 0.320 \\
\bottomrule
\end{tabular}
\end{center}

\vspace{0.3cm}
\textbf{Team totals} = 9 $\times$ individual = (26.3 R, 7.2 HR, 25.9 RBI, 48.2 SO, 80.6 TB, 3.2 SB, .320 OBP)
\end{frame}

\begin{frame}{Replacement Team vs League Average Opponent}
\textbf{League averages} (expected opponent production):
\begin{center}
\begin{tabular}{lrrr}
\toprule
Category & My Team & Opponent & Diff \\
\midrule
R & 26.28 & 28.96 & $-2.68$ \\
HR & 7.20 & 8.02 & $-0.82$ \\
RBI & 25.92 & 27.86 & $-1.94$ \\
SO & 48.24 & 50.11 & $-1.87$ (good!) \\
TB & 80.64 & 88.87 & $-8.23$ \\
SB & 3.24 & 4.74 & $-1.50$ \\
OBP & 0.320 & 0.320 & 0 \\
\bottomrule
\end{tabular}
\end{center}

\vspace{0.3cm}
A replacement-level team is \textit{below average} in most categories.

The exception: SO is below average (good, since lower is better).
\end{frame}

\begin{frame}{Win Probabilities: Replacement Team}
Using $P(\text{win}) = \Phi\left(\frac{\text{diff}}{\sigma \sqrt{2}}\right)$:

\begin{center}
\begin{tabular}{lrrrr}
\toprule
Category & Diff & SD & $z$ & $P(\text{win})$ \\
\midrule
R & $-2.68$ & 6.03 & $-0.31$ & 37.7\% \\
HR & $-0.82$ & 2.93 & $-0.20$ & 42.2\% \\
RBI & $-1.94$ & 6.72 & $-0.20$ & 41.9\% \\
SO & $+1.87$ & 7.45 & $+0.18$ & 57.0\% \\
TB & $-8.23$ & 15.94 & $-0.37$ & 35.8\% \\
SB & $-1.50$ & 2.57 & $-0.41$ & 34.0\% \\
OBP & $0$ & 0.04 & $0$ & 50.0\% \\
\bottomrule
\end{tabular}
\end{center}

\vspace{0.2cm}
\textbf{Expected category wins:} $0.377 + 0.422 + ... = 2.99$ out of 7.
\end{frame}

\begin{frame}{Enter Jos\'{e} Ram\'{i}rez}
\textbf{Ram\'{i}rez's projected season line:}\\
679 PA, 98 R, 30 HR, 94 RBI, 78 SO, 300 TB, 34 SB, .348 OBP

\vspace{0.3cm}
\textbf{Ram\'{i}rez's weekly production:}
\begin{center}
\begin{tabular}{lrrr}
\toprule
Stat & Ram\'{i}rez & Replacement & Diff \\
\midrule
R/wk & 3.92 & 2.92 & $+1.00$ \\
HR/wk & 1.20 & 0.80 & $+0.40$ \\
RBI/wk & 3.76 & 2.88 & $+0.88$ \\
SO/wk & 3.12 & 5.36 & $-2.24$ (good!) \\
TB/wk & 12.00 & 8.96 & $+3.04$ \\
SB/wk & 1.36 & 0.36 & $+1.00$ \\
OBP & 0.348 & 0.320 & $+0.028$ \\
\bottomrule
\end{tabular}
\end{center}
\end{frame}

\begin{frame}{Team After Adding Ram\'{i}rez}
Replace one replacement hitter with Ram\'{i}rez:

\begin{center}
\begin{tabular}{lrrr}
\toprule
Category & Before & After & Change \\
\midrule
R & 26.28 & 27.28 & $+1.00$ \\
HR & 7.20 & 7.60 & $+0.40$ \\
RBI & 25.92 & 26.80 & $+0.88$ \\
SO & 48.24 & 46.00 & $-2.24$ \\
TB & 80.64 & 83.68 & $+3.04$ \\
SB & 3.24 & 4.24 & $+1.00$ \\
OBP & 0.320 & 0.323 & $+0.003$ \\
\bottomrule
\end{tabular}
\end{center}

\vspace{0.3cm}
Note: OBP only shifts by $\frac{1}{9}$ of Ram\'{i}rez's OBP advantage because it's a rate stat averaged across 9 hitters.
\end{frame}

\begin{frame}{New Win Probabilities}
\begin{center}
\begin{tabular}{lrrrrr}
\toprule
Cat & Old Diff & New Diff & Old $P$ & New $P$ & $\Delta P$ \\
\midrule
R & $-2.68$ & $-1.68$ & 37.7\% & 42.2\% & $+4.5\%$ \\
HR & $-0.82$ & $-0.42$ & 42.2\% & 46.0\% & $+3.8\%$ \\
RBI & $-1.94$ & $-1.06$ & 41.9\% & 45.6\% & $+3.6\%$ \\
SO & $+1.87$ & $+4.11$ & 57.0\% & 65.2\% & $+8.1\%$ \\
TB & $-8.23$ & $-5.19$ & 35.8\% & 40.9\% & $+5.1\%$ \\
SB & $-1.50$ & $-0.50$ & 34.0\% & 44.5\% & $+10.5\%$ \\
OBP & $0$ & $+0.003$ & 50.0\% & 52.2\% & $+2.2\%$ \\
\bottomrule
\end{tabular}
\end{center}

\vspace{0.3cm}
\textbf{Ram\'{i}rez's marginal value:}
\[
\sum \Delta P = 4.5 + 3.8 + 3.6 + 8.1 + 5.1 + 10.5 + 2.2 = \mathbf{37.8\%}
\]

Ram\'{i}rez adds 0.378 expected category wins per week vs replacement.
\end{frame}

\begin{frame}{Why SB and SO Dominate}
Ram\'{i}rez's biggest impacts:
\begin{itemize}
    \item \textbf{SB: +10.5\%} from just +1.00 SB/week
    \item \textbf{SO: +8.1\%} from $-2.24$ SO/week
\end{itemize}

\vspace{0.3cm}
Compare to HR: +3.8\% from +0.40 HR/week.

\vspace{0.3cm}
\textbf{The math:}
\begin{align*}
\text{SB impact} &: +1.00 / (2.57 \times \sqrt{2}) = +0.275 \text{ z-shift} \\
\text{HR impact} &: +0.40 / (2.93 \times \sqrt{2}) = +0.097 \text{ z-shift}
\end{align*}

The steal is worth $0.275 / 0.097 = 2.8\times$ more in z-shift than the HR, despite both being ``+1 unit'' contributions (after scaling).

\vspace{0.3cm}
\textbf{Key insight:} Low absolute SD $\rightarrow$ high marginal leverage.
\end{frame}

\begin{frame}{Visualizing the Distribution Shift}
\textbf{Before Ram\'{i}rez} (SB category):
\begin{itemize}
    \item My team mean: 3.24 SB/week
    \item Opponent mean: 4.74 SB/week
    \item Difference: $-1.50$, so $z = -0.41$, $P(\text{win}) = 34\%$
\end{itemize}

\vspace{0.3cm}
\textbf{After Ram\'{i}rez}:
\begin{itemize}
    \item My team mean: 4.24 SB/week
    \item Opponent mean: 4.74 SB/week
    \item Difference: $-0.50$, so $z = -0.14$, $P(\text{win}) = 44.5\%$
\end{itemize}

\vspace{0.3cm}
The entire distribution of ``my SB $-$ opponent SB'' shifts right by 1.00 steals. Because SD is tight (2.57), this 1-steal shift moves us \textbf{10.5 percentage points} in win probability.
\end{frame}

%=============================================================================
\section{Relief Pitcher Valuation}
%=============================================================================

\begin{frame}{RP Replacement Level (Per Slot, Weekly)}
\begin{center}
\begin{tabular}{lr}
\toprule
Stat & Replacement RP \\
\midrule
IP/week & 2.48 \\
L/week & 0.118 \\
SV/week & 0.121 \\
HLD/week & 0.848 \\
K/week & 2.69 \\
ER/week & 0.96 \\
WH/week & 2.85 \\
\bottomrule
\end{tabular}
\end{center}

\vspace{0.3cm}
\textbf{Key observation:} Replacement RPs get \textbf{holds, not saves}.

Elite closers get saves but sacrifice holds. This creates an explicit tradeoff that the model captures.
\end{frame}

\begin{frame}{Why ERA/WHIP Don't Matter Much for RPs}
\textbf{ERA/WHIP are innings-weighted ratio stats.}

\vspace{0.3cm}
Typical team: 40 IP/week total
\begin{itemize}
    \item 5 SP $\times$ 6.5 IP = 32.5 IP
    \item 3 RP $\times$ 2.5 IP = 7.5 IP
\end{itemize}

\vspace{0.3cm}
A reliever contributes $\approx$\textbf{6--7\%} of team innings.

\vspace{0.3cm}
Even if an elite RP has much better ERA than replacement (e.g., 3.06 vs 3.50), the team ERA only improves by $\approx$0.02 points.

With ERA SD = 1.31, that's $0.02 / 1.31 = 0.015$ SDs $\rightarrow$ \textbf{$<$1\% win probability}.

\vspace{0.3cm}
\textbf{Conclusion:} RPs earn value through saves/holds/K, not ERA/WHIP.
\end{frame}

%=============================================================================
\section{PA Normalization Between Systems}
%=============================================================================

\begin{frame}{The Problem: Projection Systems Disagree on PT}
Different projection systems predict different playing time:

\vspace{0.3cm}
\begin{center}
\begin{tabular}{lrr}
\toprule
Player & The Bat PA & Depth Charts PA \\
\midrule
Player A & 550 & 620 \\
Player B & 480 & 510 \\
\bottomrule
\end{tabular}
\end{center}

\vspace{0.3cm}
\textbf{Problem:} We want to compare \textit{skill}, not playing time estimates.

\vspace{0.3cm}
\textbf{Solution:} Normalize all systems to use Depth Charts PA.
\end{frame}

\begin{frame}{PA Normalization Process}
\texttt{normalize\_pa.py}:

\begin{enumerate}
    \item Read Depth Charts PA for each player
    \item For each player in The Bat/BatX:
    \begin{itemize}
        \item Scale = DC\_PA / TheBat\_PA
        \item Multiply all counting stats by Scale
        \item Keep rate stats (OBP, K\%) unchanged
    \end{itemize}
    \item Output normalized CSV
\end{enumerate}

\vspace{0.5cm}
Keeping rate stats unchanged is reasonable---the rate estimates reflect the projection system's view of player skill, independent of playing time.
\end{frame}

%=============================================================================
\section{Projection System Toggle}
%=============================================================================

\begin{frame}{Three Projection Systems Available}
\begin{enumerate}
    \item \textbf{Depth Charts} (default) -- Composite of multiple systems
    \item \textbf{The Bat} -- Tom Tango's projection system
    \item \textbf{The BatX} -- Extended/experimental version
\end{enumerate}

\vspace{0.5cm}
All three use:
\begin{itemize}
    \item Same PA (normalized to Depth Charts)
    \item Same replacement level rates
    \item Same weekly SDs
\end{itemize}

\vspace{0.3cm}
\textbf{Only difference:} Rate stat projections (HR/PA, SB/PA, K\%, OBP, etc.)
\end{frame}

%=============================================================================
\section{Remaining Questions}
%=============================================================================

\begin{frame}{Open Questions for Review}
\begin{enumerate}
    \item \uncertain{Choice of ranks 155-175} -- Why not 145-165 or 160-180? The exact cutoff affects replacement level.

    \vspace{0.3cm}
    \item \uncertain{SP replacement level} -- I didn't examine this closely. Need to verify the methodology matches hitter replacement.

    \vspace{0.3cm}
    \item \uncertain{Variance equality assumption} -- The $\sqrt{2}$ assumes both teams draw from distributions with the same variance. In practice, team quality varies, but this is a reasonable simplification.
\end{enumerate}
\end{frame}

%=============================================================================
\section{Features Added (Chronological)}
%=============================================================================

\begin{frame}{Feature Timeline}
\begin{enumerate}
    \item \textbf{Base draft tool} -- UI, player data, basic rankings
    \item \textbf{Marginal win probability} -- Exact $\Phi()$ calculation
    \item \textbf{PA supplementation to 600 PA floor} -- Fill with replacement
    \item \textbf{Projection toggle} -- The Bat vs Depth Charts
    \item \textbf{PA normalization} -- All systems use DC playing time
    \item \textbf{300 PA minimum filter} -- Exclude low-PA players
    \item \textbf{Replacement recalibration} -- Ranks 155-175 cohort
    \item \textbf{OBP cap at league average} -- Prevent OBP-positive replacement
    \item \textbf{The BatX projection system} -- Third toggle option
\end{enumerate}
\end{frame}

\begin{frame}{Key Code Files}
\textbf{Python scripts:}
\begin{itemize}
    \item \texttt{create\_league\_stats.py} -- Hitter z-scores, replacement supplementation
    \item \texttt{create\_pitching\_stats.py} -- Pitcher processing
    \item \texttt{normalize\_pa.py} -- Align PA across systems
\end{itemize}

\vspace{0.5cm}
\textbf{Main application:}
\begin{itemize}
    \item \texttt{draft\_tool.html} -- All-in-one HTML/CSS/JS app
    \begin{itemize}
        \item Lines 482-506: SD and replacement constants
        \item Lines 580-599: \texttt{normalCDF} and \texttt{winProbability}
        \item Lines 604+: Roster projection calculations
    \end{itemize}
\end{itemize}

\vspace{0.5cm}
\textbf{Data files:}
\begin{itemize}
    \item \texttt{fantasy\_hitters\_dc\_2026.csv}, \texttt{fantasy\_hitters\_thebat\_2026.csv}, \texttt{fantasy\_hitters\_batx\_2026.csv}
\end{itemize}
\end{frame}

\begin{frame}{Summary}
\textbf{What the tool does:}
\begin{itemize}
    \item Calculates marginal win probability for each player vs replacement
    \item Uses weekly SDs from 2024 league data to weight categories
    \item Normalizes PA across projection systems for fair comparison
    \item Supplements low-PA players to 600 PA baseline
\end{itemize}

\vspace{0.5cm}
\textbf{Key insight:} Categories with low absolute SD (SB, SV, HLD) have outsized marginal impact. A single steal shifts win probability more than a single HR because $\text{SD}_{SB} < \text{SD}_{HR}$.
\end{frame}

\end{document}
